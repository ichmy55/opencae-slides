\begin{frame}[noframenumbering]{付録B. ~配布するデータファイル~}
  今回使用したファイルについては以下GitHub
     ({\urlstyle{same} \color{cud_orange}
      \href{https://github.com/ichmy55/opencae-slides/releases/tag/v.0.1.2/handout.zip}
      {github.com/ichmy55/opencae-slides/releases/tag/v.0.1.2/handout.zip}})にて配布します
   \begin{table}[hbtp]
    \caption{配布zipの中身}
    \vspace{-5mm}
   {\footnotesize
      \begin{tabular}{|c||l|l|l|} \hline % 表は項目名を中央寄せ、データを左寄せ
        ディレクトリ & ファイル種類 & ファイル名 & 概要 \\ \hhline{|=:=|=|=|}
	\multirow{2}{*}{01-gmesh-script} & gmsh入力  & failure-example.geo & メッシュ切失敗形状  \\ \cline{3-4}
					 & スクリプト& successful-example.geo & メッシュ切成功形状  \\ \hline
	      \multirow{2}{*}{02-openCASCADE-brep} & gmsh出力  & failure-example.brep & メッシュ切失敗形状  \\ \cline{3-4}
                                         & B-rep形式& successful-example.brep & メッシュ切成功形状  \\ \hline
      \end{tabular}
    }
  \end{table}
\end{frame}
