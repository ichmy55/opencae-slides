\documentclass[               %
  unicode,                    % 文字コードの指定
  mathserif,                  %
  m,                          % vertical position of text: (t,m,b)
  % handout,                  % for printing slides
  aspectratio=169,            % 画面サイズを(16:9)にする
  12pt,                       % フォント基本サイズを12ptにします
  % compress,
  unknownkeysallowed
]{beamer}
%
% 各種設定用ファイルの読み込み:ここのファイル名は決め打ちにさせてください
%
  %
% パッケージの選択
%
\usepackage{luatexja}% 日本語に
\usepackage[haranoaji,deluxe]{luatexja-preset}% フォント指定
\renewcommand{\kanjifamilydefault}{\gtdefault}% 既定をゴシック体に
\usepackage{url}              % LaTeXの文章内にURLを貼りたいとき
\usepackage{hyperref}         % TeX 文書(DVI、PDF など)に HTML と同じハイパーリンク 機能を加えるためのマクロ
\usepackage{siunitx}          % LATEX で SI単位(国際単位系)を出力する
\usepackage{enumitem}         % リスト環境のレイアウトを制御
\usepackage{tikz}             % TeX 用の描画パッケージ
\usepackage{
    amsmath, % 環境
    amssymb, % 記号
    amsfonts % 特殊文字
}                             % amsmathの数式環境
\usetikzlibrary{positioning}  % ”positioning”ライブラリ
%
% デザインの選択
%     ここではテーマとしてmetropolisを読込
%
\usetheme[
    block=fill, % ブロックに背景をつける
    numbering=fraction % 合計ページ数を表示
]{metropolis}

%
% ここからはページの見え方の設定
%
% ページ番号
\setbeamerfont{frame numbering}{size=\large}
%
% CUD 配色の作成
% cf. http://bit.ly/2G99WCG
\definecolor{cud_blue}{rgb}{.109803922,.349019608,.682352941}
\definecolor{cud_green}{rgb}{.282352941,.639215686,.407843137}
\definecolor{cud_orange}{rgb}{.929411765,.564705882,.156862745}
\definecolor{cud_lightgray}{rgb}{.784313725,.784313725,.796078431}
%
% 基本色の変更
\setbeamercolor{normal text}{fg=cud_blue}
\setbeamercolor{example text}{fg=cud_green}
\setbeamercolor{alerted text}{fg=cud_orange}
%
% Navigation symbol は不要なので消す
\setbeamertemplate{navigation symbols}{}
%
%footer修正
\def\logoC{opencae-logo.png}
\def\logoD{kotohajime.png}
\makeatletter
\setbeamertemplate{footline}{
    \begin{columns}[totalwidth=160mm]
      \begin{column}{24.1mm}
        \includegraphics[width=24.1mm, height=5mm]{work/images/\logoC}
      \end{column}
      \begin{column}{104mm}
         {\scriptsize \color{cud_lightgray} \insertshortinstitute / \insertshortdate{}  (\insertshortauthor) }
      \end{column}
      \begin{column}{16mm}
         {\footnotesize \color{cud_orange} \rightline{ \insertframenumber{} / \inserttotalframenumber}}
      \end{column}
      \begin{column}{15mm}
        \includegraphics[width=15mm, height=5mm]{work/images/\logoD}
      \end{column}
    \end{columns}
}
   % 資料毎に変えない設定を読み込む
  % 表題の例
%     title     に主題
%     subtittle に副題を設定(省略可)
%
\title{圧力容器の応力値を手計算とCAEで比較した}
\subtitle{~ PrePoMaxは、やればできる子~}
%
% 日付/発表者設定の例
%     date      に発表日の日付
%     author    に発表者の名前
%     institute には本来は発表者の所属先ですが、下の例では発表会の名前とした
%     それぞれ左の角カッコ内が簡易表記で、表紙以外で使われ
%     右の波カッコ内が詳細表記で、表紙で使われる
%
\date[2-24-2024]{発表日:2024年2月24日}
\author[ichmy55]{発表者:ichmy55}
\institute[  25th Meeting/OpenCAE Local User Group@Kantou(str)]{報告:第25回オープンCAE勉強会@関東(構造など)}
  % 資料毎に変える設定を読み込む
%
% ここから、各章毎に分割したソースファイルを順番に読込
%
\begin{document}
  %
  \maketitle                         % 表紙
  \begin{frame}{はじめに}
  \begin{itemize}[itemsep=2.5ex, leftmargin=3mm]
      \large
      \item[〇] 報告者がCAEを始めたて初心者の時点で(商用コードで)出来て

                いたことを、OpenCAEを用いて再現した

      \item[〇] 上記のうち、メッシュ切について日本語で説明している資料が少ないのが残念

      \item[〇] 最近OpenCAEでも簡単に切れるようになった6面体

                メッシュを 中心に、まとめてみた
  \end{itemize}
\end{frame}
          % はじめに
  \begin{frame}{本日の流れ}
  \begin{itemize}
     \item[▶] \highlight[cud_yellow]{目次}
     \begin{itemize}[itemsep=1.3ex, leftmargin=1cm]
       \item[1.] 自己紹介
       \item[2.] 報告者が当時できていたこと
       \item[3.] 本日の例題と以前の勉強会で報告された結果
       \item[4.] あらためて解いてみた
       \item[5.] まとめ
    \end{itemize}
  \end{itemize}
\end{frame}
           % 目次
  %%%
  %%%
  \begin{frame}{本日の流れ}
  \begin{itemize}
     \item[] 目次
     \begin{itemize}[itemsep=1.3ex, leftmargin=1cm]
        \item[▶1.] \highlight[cud_yellow]{ 自己紹介}
        \item[2.] OpenCAEの構造解析でもメッシュにこだわりたい
        \item[3.] 本日の例題と以前の計算結果
        \item[4.] 改めて解いてみた(PrePoMaxは、やればできる子)
        \item[5.] まとめ
        \item[A.] 付録 ~ソースのありか~
     \end{itemize}
  \end{itemize}
\end{frame}
           % 
  ../opencae-kansai-099/011-self-introduce.tex    % 自己紹介
  %%%
  \begin{frame}{本日の流れ}
  \begin{itemize}
      \item[] 目次
      \begin{itemize}[itemsep=1.3ex, leftmargin=1cm]
        \item[1.]  {\color{cud_lightgray} 自己紹介}
        \item[▶2.] \highlight[cud_yellow]{ OpenCAEの構造解析でもメッシュにこだわりたい }
        \item[3.] 本日の例題と以前の勉強会で報告された結果
        \item[4.] 改めて解いてみた(PrePoMaxは、やればできる子)
        \item[5.] まとめ
     \end{itemize}
  \end{itemize}
\end{frame}
           % 
  \begin{frame}{報告者が当時出来ていたこと}
  \begin{table}[hbtp]
      \caption{報告者が当時出来ていたこと}
      \begin{tabular}{|r|l|} % 表は項目名を右寄せ、データを左寄せ
          \hline
          形状     & 3次元形状のみ \rule[0mm]{0mm}{7mm} \\
          \hline
          メッシュ & 主に4面体、一部6面体 \rule[0mm]{0mm}{7mm} \\
          \hline
          境界条件 & 変位拘束と荷重(分布or集中) \rule[0mm]{0mm}{7mm} \\
          \hline
          結果処理 & 反力のチェックサム \rule[0mm]{0mm}{7mm} \\
                   & 接点データ外だし→表計算ソフトでグラフ化\\
          \hline
    \end{tabular}
  \end{table}
\end{frame}
      % 
  %%%
  \begin{frame}{本日の流れ}
  \begin{itemize}
      \item[] 目次
      \begin{itemize}[itemsep=1.3ex, leftmargin=1cm]
        \item[1.] {\color{cud_lightgray} 自己紹介}
        \item[2.] {\color{cud_lightgray} 報告者が当時出来ていたこと}
        \item[▶3.] \highlight[cud_yellow]{ 本日の例題と以前の勉強会で報告された結果}
        \item[4.] あらためて解いてみた
        \item[5.] まとめ
     \end{itemize}
  \end{itemize}
\end{frame}
           % 
  \begin{frame}{本日の例題}
 
    \begin{columns}[t]
    \begin{column}{0.65\textwidth}
        \\
        <参考文献\cite{wanted} より例題を拝借> \\
        右\figurename \ref{fig:example-probrem}に示すような圧力容器に内圧10MPaが \\
        かかっている。
        材料はSB450で、ヤング率は205GPa、降伏応力は250MPaである。 \\
        本構造は薄肉構造と近似できるとする。
      \begin{itemize}
          \item{円筒部のA点と半球部のB点の応力状態を求めよ}
          \item{A点とB点ミーゼス相当応力を求め、幸福応力に達する臨界圧力を求めよ}
      \end{itemize}
    \end{column}
    \begin{column}{0.35\textwidth}
      \begin{figure}[htbp]
        \begin{center}
          \includegraphics[keepaspectratio,scale=2.2]{work/images/example-probrem.png}
            \caption{本日の例題(圧力容器)} \label{fig:example-probrem}
        \end{center}
      \end{figure}
    \end{column}
  \end{columns}

% 図の挿入
%\begin{figure}[htbp]
%\begin{center}
%\includegraphics[keepaspectratio,scale=1.0]{work/images/fig01.jpg}
%\caption{4面体メッシュ切り(成功例)}
%\end{center}
%\end{figure}
 
\end{frame}
   % 本日の例題と以前の勉強会で報告された結果
  \begin{frame}{以前の勉強会で報告された結果}
 
    \begin{columns}[t]
    \begin{column}{0.7\textwidth}
       <報告者コメント> \\
         ミーゼス応力で比較した結果、A点応力に差異 \\
          \begin{table}[hbtp]
            \begin{tabular}{rlp{10em}} % 表は項目名を右寄せ、データを左寄せ
               手計算 & 52 [\si{\mega\pascal}] & \\
               CAE    & 58.9 [\si{\mega\pascal}]  & \\
            \end{tabular}
          \end{table}
        <参加者コメント(一部のみ抜粋し要約)> \\
         \begin{itemize}
            \item[①] コンター図がまだら模様でおかしい。\\
                     \Add{メッシュ}に問題がある。 \\
                     正しく計算したければ \highlight[cud_yellow]{6面体メッシュ}で厚み\\
                     方向に4層切りメッシュを作る必要がある。\\
            \item[②] CAE結果と比較する相手として、薄肉構造を仮定 \\
                     % textlint-disable
                     した手計算は相応しくない?
                     % textlint-enable 
         \end{itemize}
    \end{column}
    \begin{column}{0.3\textwidth}
      \begin{figure}[htbp]
        \begin{center}
          \includegraphics[keepaspectratio,scale=1.5]{work/images/previous.png}
            \caption{以前の勉強会で報告された結果} \label{fig:previous}
        \end{center}
      \end{figure}
    \end{column}
  \end{columns}
  %
  % TiKZを使った図形の描画 図2でA点を指し示す矢印
  \begin{textblock*}{30pt}(385pt,95pt)
    \begin{tikzpicture}
        \draw[->, draw=cud_red, line width=2pt] (0.7,0.8) -- (0,0.3);
        \node[rectangle,fill=cud_yellow,text width=0.5cm,text centered,rounded corners,minimum height=0.5cm](s) at (1cm,1cm) { \scriptsize A点};
    \end{tikzpicture}
  \end{textblock*}
\end{frame}
 % 以前の勉強会でのコメント
  %%%
  \begin{frame}{本日の流れ}
  \begin{itemize}
      \item[] 目次
    \begin{enumerate}[label=\textbf{ \arabic*.},itemsep=1.3ex, leftmargin=1cm]
        \item[1.] 自己紹介
        \item[2.] OpenCAEの構造解析でもメッシュにこだわりたい
        \item[3.] 本日の例題
        \item[▶4.] \highlight[cud_yellow]{ 以前の計算結果と指摘 }
        \item[5.] 改めて解いてみた(PrePoMaxは、やればできる子)
        \item[6.] まとめ
        \item[A.] 付録 ~ソースのありか~
    \end{enumerate}
  \end{itemize}
\end{frame}
           % 改めて解いてみた(PrePoMaxは、やればできる子)
  \input{work/041-improvements}      % 今回の改善点
  \input{work/042-meshing}           % 改めてメッシュについて復習
  \input{work/043-hex8}              % 
  %%%
  \begin{frame}{本日の流れ}
  \begin{itemize}
      \item[] 目次
      \begin{itemize}[itemsep=1.3ex, leftmargin=1cm]
        \item[1.] 自己紹介
        \item[2.] OpenCAEの構造解析でもメッシュにこだわりたい
        \item[3.] 本日の例題と以前の計算結果と指摘
        \item[4.] 改めて解いてみた(PrePoMaxは、やればできる子) 
        \item[▶5.] \highlight[cud_yellow]{ まとめ }
        \item[A.] 付録 ~ソースのありか~
      \end{itemize}
  \end{itemize}
\end{frame}
           % 
  \begin{frame}{まとめ}
  \begin{itemize}[itemsep=2.5ex, leftmargin=3mm]
      \large
      \item[〇] PrePoMaxは、初級の構造解析者が使えるべき機能を \\
                十分に備えている

      \item[〇] 今回の報告では特にメッシュ切を中心にまとめた

      \item[〇] 構造解析者の入門用としての普及に協力し \\
                構造解析者の後輩が増える一助になりたい

  \end{itemize}
\end{frame}
       % まとめ
  %%%
  \begin{frame}{4面体メッシュ切り(成功例)}
 
% 図の挿入
\begin{figure}[htbp]
\begin{center}
\includegraphics[keepaspectratio,scale=1.0]{work/images/fig01.jpg}
\caption{4面体メッシュ切り(成功例)}
\end{center}
\end{figure}
 
\end{frame}

  \input{work/062-chapter6-2}
  \input{work/063-chapter6-3}
  %%%
  \begin{frame}{参考文献}
  \begin{enumerate}[label=\textbf{[\arabic*]},itemsep=1ex, leftmargin=1cm]
    \item BannerKoubou.com.“無料で使えるバナー作成素材”.https://bannerkoubou.com/banner/ \\
          本資料中の左下部のバナーはこちらで作成しました
  \end{enumerate}
\end{frame}
        % 参考文献
  \begin{frame}[noframenumbering]{}
	ご清聴、ありがとうございました
\end{frame}
           % ご清聴、ありがとうございました
  \begin{frame}[noframenumbering]{付録A. ~ソースのありか~}
  \begin{table}[hbtp]
    \begin{tabular}{rll} % 表は項目名を右寄せ、データを左寄せ
      項目                & 置き場 & URL \\
        スライド.tex  & \multirow{3}{*}{Github} &  \multirow{3}{*}{\urlstyle{same} \color{cud_orange}
                                   \href{https://github.com/ichmy55/opencae-slides}
                                        {github.com/ichmy55/opencae-slides}} \\
        形状.geo  & & \\
        形状.step & & \\
        図面      &  Onshape & 予定地 \\
        スライド.pdf  & Docswell & {\urlstyle{same} \color{cud_orange}
                                   \href{https://www.docswell.com/user/ichmy55}
                                   {www.docswell.com/user/ichmy55}} \\
    \end{tabular}
  \end{table}
\end{frame}
      % 付録 ~ソースのありか~
\end{document}
