% 表題の例
%     title     に主題
%     subtittle に副題を設定(省略可)
%
\title{6面体メッシュのすすめ}
\subtitle{~ PrePoMaxは、やればできる子~}
%
% 日付/発表者設定の例
%     date      に発表日の日付
%     author    に発表者の名前
%     institute には本来は発表者の所属先ですが、下の例では発表会の名前とした
%     それぞれ左の角カッコ内が簡易表記で、表紙以外で使われ
%     右の波カッコ内が詳細表記で、表紙で使われる
%
%     subject(PDFinfoではサブタイトル) と keywords  にも設定しますが、
%     生成PDFの info に追記されるもので、画面上には表示されません
%
\date[4-22-2024]{報告日:2024年4月22日}
\author[ichmy55]{報告者:ichmy55}
\institute[  99th Meeting/OpenCAE Local User Group@Kansai]{報告:第99回オープンCAE勉強会@関西}
\subject{Usage example of PrePoMax}
\keywords{PrePoMax,OpenCAE}
