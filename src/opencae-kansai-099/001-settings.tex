%
% パッケージの選択
%
\usepackage{luatexja}% 日本語に
\usepackage[haranoaji,deluxe]{luatexja-preset}% フォント指定
\renewcommand{\kanjifamilydefault}{\gtdefault}% 既定をゴシック体に
\usepackage{url}              % LaTeXの文章内にURLを貼りたいとき
\usepackage{hyperref}         % TeX 文書(DVI、PDF など)に HTML と同じハイパーリンク 機能を加えるためのマクロ
\usepackage{siunitx}          % LATEX で SI単位(国際単位系)を出力する
\usepackage{enumitem}         % リスト環境のレイアウトを制御
\usepackage{tikz}             % TeX 用の描画パッケージ
\usepackage{
    amsmath, % 環境
    amssymb, % 記号
    amsfonts % 特殊文字
}                             % amsmathの数式環境
\usetikzlibrary{positioning}  % ”positioning”ライブラリ
%
% デザインの選択
%     ここではテーマとしてmetropolisを読込
%
\usetheme[
    block=fill, % ブロックに背景をつける
    numbering=fraction % 合計ページ数を表示
]{metropolis}

%
% ここからはページの見え方の設定
%
% ページ番号
\setbeamerfont{frame numbering}{size=\large}
%
% CUD 配色の作成
% cf. http://bit.ly/2G99WCG
\definecolor{cud_blue}{rgb}{.109803922,.349019608,.682352941}
\definecolor{cud_green}{rgb}{.282352941,.639215686,.407843137}
\definecolor{cud_orange}{rgb}{.929411765,.564705882,.156862745}
\definecolor{cud_lightgray}{rgb}{.784313725,.784313725,.796078431}
%
% 基本色の変更
\setbeamercolor{normal text}{fg=cud_blue}
\setbeamercolor{example text}{fg=cud_green}
\setbeamercolor{alerted text}{fg=cud_orange}
%
% Navigation symbol は不要なので消す
\setbeamertemplate{navigation symbols}{}
%
%footer修正
\def\logoC{opencae-logo.png}
\def\logoD{kotohajime.png}
\makeatletter
\setbeamertemplate{footline}{
    \begin{columns}[totalwidth=160mm]
      \begin{column}{24.1mm}
        \includegraphics[width=24.1mm, height=5mm]{work/images/\logoC}
      \end{column}
      \begin{column}{104mm}
         {\scriptsize \color{cud_lightgray} \insertshortinstitute / \insertshortdate{}  (\insertshortauthor) }
      \end{column}
      \begin{column}{16mm}
         {\footnotesize \color{cud_orange} \rightline{ \insertframenumber{} / \inserttotalframenumber}}
      \end{column}
      \begin{column}{15mm}
        \includegraphics[width=15mm, height=5mm]{work/images/\logoD}
      \end{column}
    \end{columns}
}
