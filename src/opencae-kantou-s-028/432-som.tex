\begin{frame}{今回使用するソフトと選定理由}
 \begin{table}[hbtp]
    \caption{今回使用するソフト}
    \begin{NiceTabular}{|c|c|c|c|c|c|} % 表は項目名を右寄せ、データを中寄せ
       \hline
	    用途     &  形状作成    & メッシュ切 & プリ & ソルバー & ポスト \\
       \hline
	    表で動作 &  gmsh        & \multicolumn{4}{|c|}{PrePoMax} \\
       \hline
	    内部動作 &  OpenCascade & gmsh       & PrePoMax  & Calculix & VTK \\
       \hline
    \end{NiceTabular}
  \end{table}
  \begin{itemize}
     \item[①] gmsh/OpenCascade :\\
              マルチブロックをまとめて一つのグループとして扱い \\
              ブロック間のメッシュの割れを防ぐ機能がある \\
              作成経過がスクリプト化され、形状の公開/後日の検証がしやすい
     \item[②] PrePoMax :
              とっつきやすそう
     \item[③] Calculix :
              上記の同梱だから
  \end{itemize}
\end{frame}
