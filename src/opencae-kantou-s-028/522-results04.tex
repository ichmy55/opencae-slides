\begin{frame}{接点応力データの取り出し}
  \begin{textblock*}{0.95\linewidth}(0pt,5pt)
  \begin{figure}[htbp]
    \begin{center}
      % TiKZを使った図形の描画
      \begin{tikzpicture}
	      \node [above right,minimum height=50pt, minimum width=420pt,align=left] at (0pt,20pt) {calculixでは、接点応力データは*.frd ファイルに、接点の選択情報は *.dat \\
	  に出力されるが、これらのファイルは単なるアスキーの数字の羅列ですので、\\
	  メモ帳で必要な行だけ取り出し、表計算ソフトで処理する};
         \node [draw=blue,below right,minimum height=100pt,minimum width=80pt, align=center,fill=cud_lightpink] at (20pt,20pt){全接点情報 \\ *.frd};
         \node [draw=blue,below right,minimum height=40pt,minimum width=80pt,align=center,fill=cud_lightpink] at (20pt,-85pt){指定接点情報 \\ *.dat};
         \node [draw=blue,below right,minimum height=20pt,minimum width=100pt,align=center,fill=cud_lightgray] at (150pt,20pt){接点座標.txt};
         \node [draw=blue,below right,minimum height=20pt,minimum width=100pt,align=center,fill=cud_lightgray] at (150pt,-5pt){要素構成接点.txt};
         \node [draw=blue,below right,minimum height=20pt,minimum width=100pt,align=center,fill=cud_lightgray] at (150pt,-30pt){接点変位.txt};
         \node [draw=blue,below right,minimum height=20pt,minimum width=100pt,align=center,fill=cud_lightgray] at (150pt,-55pt){接点応力.txt};
         \node [draw=blue,below right,minimum height=20pt,minimum width=100pt,align=center,fill=cud_lightgray] at (150pt,-95pt){対象接点.txt};
         \draw[very thick,->] (100pt,10pt)--(150pt,10pt);
         \draw[very thick,->] (100pt,-15pt)--(150pt,-15pt);
         \draw[very thick,->] (100pt,-40pt)--(150pt,-40pt);
         \draw[very thick,->] (100pt,-65pt)--(150pt,-65pt);
         \draw[very thick,->] (100pt,-105pt)--(150pt,-105pt);
         \node [below right,minimum height=20pt,minimum width=100pt,align=center,font=\small] at (75pt,10pt){メモ帳};
         \node [draw=blue,below right,minimum height=140pt,minimum width=30pt, align=center] at (300pt,20pt){表\\計\\算\\ソ\\フ\\ト};
         \draw[very thick,->] (250pt,10pt)--(300pt,10pt);
         \draw[very thick,->] (250pt,-15pt)--(300pt,-15pt);
         \draw[very thick,->] (250pt,-40pt)--(300pt,-40pt);
         \draw[very thick,->] (250pt,-65pt)--(300pt,-65pt);
         \draw[very thick,->] (250pt,-105pt)--(300pt,-105pt);
         \node [below right,minimum height=20pt,minimum width=100pt,align=center,font=\scriptsize] at (225pt,10pt){テキスト\\読込};
         \node [draw=blue,below right,minimum height=140pt,minimum width=30pt, align=center,fill=cud_yellow] at (380pt,20pt){グ\\ラ\\フ\\化};
         \draw[very thick,->] (330pt,-40pt)--(380pt,-40pt);
         \node [below right,minimum height=20pt,minimum width=100pt,align=center,font=\small] at (300pt,-5pt){座標\\変換};
      \end{tikzpicture}
      \caption{接点応力データの取り出し}
    \end{center}
  \end{figure}
	  \end{textblock*}
\end{frame}
