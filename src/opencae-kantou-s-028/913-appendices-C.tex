\begin{frame}[noframenumbering]{付録C. ~配布するデータファイル~}
  前ページで示した配布物の中身は以下のとおり
   \begin{table}[hbtp]
    \caption{Handout.zipの中身}
    \vspace{-2mm}
   {\scriptsize
      \begin{tabular}{|c||l|l|l|} \hline % 表は項目名を中央寄せ、データを左寄せ
        ディレクトリ & ファイル種類 & ファイル名 & 概要 \\ \hhline{|=:=|=|=|}
	\multirow{2}{*}{01-gmsh-script} & gmsh入力  & failure-example.geo & メッシュ切失敗形状  \\ \cline{3-4}
				        & スクリプト& successful-example.geo & メッシュ切成功形状  \\ \hline
        \multirow{4}{*}{02-gmsh-shape}  & gmsh形状出力  & failure-example.brep & メッシュ切失敗形状  \\ \cline{3-4}
				        & B-rep形式 & successful-example.brep & メッシュ切成功形状  \\ \cline{2-4}
                                        & gmsh形状出力  & failure-example.step & メッシュ切失敗形状  \\ \cline{3-4}
                                        & step形式  & successful-example.step & メッシュ切成功形状  \\ \hline
        03-prepomax-savefile            & prepomaxセーブ  & success-example-hex8.pmx &   \\ \hline
        04-calculix-input               & calculix入力 & success-example-hex8.inp &   \\ \hline
   \multirow{2}{*}{05-caluculix-output} & \multirow{2}{*}{calculix出力}  & success-example-hex8.frd & 全接点応力出力  \\ \cline{3-4}
				        & & success-example-hex8.dat & 特定接点変位出力  \\ \hline
        06-libreoffice                  & 表計算 & successful-example-hex8.ods &   \\ \hline
      \end{tabular}
    }
  \end{table}
\end{frame}
