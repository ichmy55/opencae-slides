\begin{frame}{6面体メッシュが切れる形状の条件の復習}
  %
   \begin{columns}[t]
    \begin{column}{0.7\textwidth}
      <6面体メッシュが切れる形状の条件>
      \begin{itemize}
        \item[(1)] 4辺形6面で構成 \\
	           トポロジー的に直方体と等しいこと \\
		   (1面が縮退して5面体(プリズム状)も可能)
	\item[(2)] それぞれの辺/面は曲がっていてもよい
	\item[(3)] ただし、品質の良いメッシュを切るには、\\
		   角が90度に近いほうがよい
      \end{itemize}
    \end{column}
    \begin{column}{0.3\textwidth}
    \end{column}
  \end{columns}
\end{frame}
