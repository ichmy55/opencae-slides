\begin{frame}[noframenumbering]{付録A. ~今回使用した主なソフト~}
  \begin{table}[hbtp]
    \caption{今回使用した主なソフト}
    \vspace{-7mm}
    \begin{tabular}{|c||l|l|l|} \hline % 表は項目名を中央寄せ、データを左寄せ
	    用途          & 名称 & バージョン & URL \\ \hhline{|=:=|=|=|}
      3次元形状作成 & gmsh & 4.13.1 & {\urlstyle{same} \color{cud_orange}
                                   \href{https://gmsh.info}
				   {gmsh.info}}  \\ \hline
      プリ・ポスト  & PrePoMax & 2.1.7 & {\urlstyle{same} \color{cud_orange}
                                   \href{https://prepomax.fs.um.si/}
				   {prepomax.fs.um.si}}  \\ \hline
      ソルバ(PrePoMax同梱) & Calculix & 2.21 & {\urlstyle{same} \color{cud_orange}
                                   \href{https://calculix.de/}
				   {calculix.de}}  \\ \hline
      表計算        & LibreOffice & 24.8.0.3 & {\urlstyle{same} \color{cud_orange}
                                   \href{https://libreoffice.org/}
				   {libreoffice.org}}  \\ \hline
      OS            & Windows & 11Pro(23H2) & {\urlstyle{same} \color{cud_orange}
                                   \href{https://www.microsoft.com/}
				   {microsoft.com}}  \\ \hline
    \end{tabular}
    \\(注:各ソフトはWindows版x86\_64/amd64用を使用しています)
  \end{table}
\end{frame}
