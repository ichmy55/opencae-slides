\begin{frame}{報告者が当時できていたこと}
  \begin{table}[hbtp]
      \caption{報告者が当時できていたこと}
      \begin{tabular}{|r|l|} % 表は項目名を右寄せ、データを左寄せ
          \hline
          形状     & 3次元形状のみ \rule[0mm]{0mm}{7mm} \\
          \hline
          材質     & 線形弾性体のみ \rule[0mm]{0mm}{7mm} \\
          \hline
          メッシュ & 主に4面体、一部 \highlight[cud_lightpink]{6面体}\rule[0mm]{0mm}{7mm} \\
                   &  \\
          \hline
          境界条件 & 変位拘束と荷重(分布or集中) \rule[0mm]{0mm}{7mm} \\
          \hline
          結果処理 & 反力のチェックサム \rule[0mm]{0mm}{7mm} \\
                   & 接点データ外だし→表計算ソフトでグラフ化 \\
          \hline
          \multicolumn{2}{c}{これらをOpenCAEで試してみる}  \rule[0mm]{0mm}{7mm}
    \end{tabular}
  \end{table}
\end{frame}
