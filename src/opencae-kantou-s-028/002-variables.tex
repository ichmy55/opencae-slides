% 表題
%     title     に主題
%     subtittle に副題
%
\title{6面体メッシュのすすめ}
\subtitle{~ 前々回でした質問の件、自分なりに調査しました ~}
%
% 日付/発表者設定
%     date      に発表日の日付
%     author    に発表者の名前
%     institute には本来は発表者の所属先ですが、下の例では発表会の名前とした
%     それぞれ左の角カッコ内が簡易表記で、表紙以外で使われ
%     右の波カッコ内が詳細表記で、表紙で使われる
%
%     subject(PDFinfoではサブタイトル) と keywords  にも設定するが
%     生成PDFの info に追記されるもので、画面上には表示されない
%
\date[Dec.14th,2024]{報告日:2024年12月14日}
\author[ichmy55]{報告者:ichmy55}
\institute[  28th Meeting/OpenCAE Local User Group@Kantou]{報告:第28回オープンCAE勉強会@関東(構造など)}
\subject{Recommendation for hexahedral mesh}
\keywords{hexa mesh,PrePoMax,OpenCAE}
