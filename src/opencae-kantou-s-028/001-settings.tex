%
% パッケージの選択
%
\usepackage{luatexja}% 日本語に
\usepackage[haranoaji,deluxe]{luatexja-preset}% フォント指定
\renewcommand{\kanjifamilydefault}{\gtdefault}% 既定をゴシック体に
\usepackage{url}              % LaTeXの文章内にURLを貼りたいとき
\usepackage{hyperref}         % TeX 文書(DVI、PDF など)に HTML と同じハイパーリンク 機能を加えるためのマクロ
\usepackage{siunitx}          % LATEX で SI単位(国際単位系)を出力する
\usepackage{dcolumn}          % 小数点揃え
\usepackage{enumitem}         % リスト環境のレイアウトを制御
\usepackage{booktabs}         % 表作成
\usepackage{multirow}         % 表作成時に表を縦結合する
\usepackage{hhline}           % 表作成時に2重枠線
\usepackage{ulem}             % 打消し線などの文字装飾
\usepackage{tikz}             % TeX 用の描画パッケージ
\usepackage{
    amsmath, % 環境
    amssymb, % 記号
    amsfonts % 特殊文字
}                             % amsmathの数式環境
\usetikzlibrary{positioning}     % ”positioning”ライブラリ
\usetikzlibrary{shapes.callouts} % 吹き出しの形を使うためにshapesライブラリを利用します
\usepackage[absolute,overlay]{textpos} % textposパッケージ
%\usepackage[colorgrid,gridunit=pt,texcoord]{eso-pic}
             % eso-picパッケージを使ってスライドの上からグリッドを表示、リリース時は消しておく
%\usepackage{layout}
             % layout パッケージもリリース時は消す
%
% デザインの選択
%     ここではテーマとしてmetropolisを読込
%
\usetheme[
    block=fill, % ブロックに背景をつける
    numbering=fraction % 合計ページ数を表示
]{metropolis}

%
% ここからはページの見え方の設定
%
% ページ番号
\setbeamerfont{frame numbering}{size=\large}
%
% CUD 配色の作成
% cf. http://bit.ly/2G99WCG
\definecolor{cud_blue}{rgb}{.109803922,.349019608,.682352941}
\definecolor{cud_green}{rgb}{.282352941,.639215686,.407843137}
\definecolor{cud_orange}{rgb}{.929411765,.564705882,.156862745}
\definecolor{cud_red}{rgb}{1.,.156862745,0.}
\definecolor{cud_yellow}{rgb}{.98039215686,.96078431372., .0}
\definecolor{cud_lightgray}{rgb}{.784313725,.784313725,.796078431}
\definecolor{cud_lightpink}{rgb}{1.,.819607843,.819607843}
%
% 基本色の変更
\setbeamercolor{normal text}{fg=cud_blue}
\setbeamercolor{example text}{fg=cud_green}
\setbeamercolor{alerted text}{fg=cud_orange}
%
% Navigation symbol は不要なので消す
\setbeamertemplate{navigation symbols}{}
%
% 見出しのスペースを消したい
\setbeamertemplate{frametitle}{
  \nointerlineskip
  \begin{beamercolorbox}[wd=\paperwidth,ht=2.25ex,dp=0.75ex]{frametitle} % htで直接指定
    \hspace*{1ex}\insertframetitle % 左margin + hspaceから始める
  \end{beamercolorbox}
}
%
% 図表番号を「図1」「表1」とかにする
\renewcommand{\figurename}{図}
\renewcommand{\tablename}{表}
\setbeamertemplate{caption}[numbered]
%
% 図とキャプションの間の余白
\setlength\abovecaptionskip{0pt}
\setlength\belowcaptionskip{0pt}
%
% highlight マクロ tasusu.github.io/tikz.html から頂きました
\newcommand{\highlight}[2][yellow]{\tikz[baseline=(x.base)]{\node[rectangle,rounded corners,fill=#1](x){#2};}}
\newcommand{\highlightcap}[3][yellow]{\tikz[baseline=(x.base)]{\node[rectangle,rounded corners,fill=#1!10](x){#2} node[below of=x, color=#1]{#3};}}
%
% Erase マクロ lilaboc.work/archives/17240764.html から頂きました
\newcommand{\Add}[1]{\textcolor{cud_red}{#1}}
\newcommand{\Erase}[1]{\textcolor{cud_red}{\sout{\textcolor{cud_blue}{#1}}}}
%
%footer修正
\def\logoC{opencae-logo.png}
\def\logoD{by.png}
\makeatletter
\setbeamertemplate{footline}{
    \begin{columns}[totalwidth=160mm]
      \begin{column}{24.1mm}
        \includegraphics[width=24.1mm, height=5mm]{work/images/\logoC}
      \end{column}
      \begin{column}{104mm}
         {\scriptsize \color{cud_lightgray} \insertshortinstitute / \insertshortdate{}  (\insertshortauthor) }
      \end{column}
      \begin{column}{16mm}
         {\footnotesize \color{cud_green} \rightline{ \insertframenumber{} / \inserttotalframenumber}}
      \end{column}
      \begin{column}{15mm}
        \includegraphics[width=15mm, height=5mm]{work/images/\logoD}
      \end{column}
    \end{columns}
}
