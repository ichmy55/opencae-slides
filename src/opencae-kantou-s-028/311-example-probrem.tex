\begin{frame}{本日の例題}
 
    \begin{columns}[t]
    \begin{column}{0.65\textwidth}
        \\
        <参考文献\cite{wanted} より例題を拝借> \\
	    右\figurename \ref{fig:example-probrem}に示すような圧力容器に内圧10[\si{\mega\pascal}]が \\
        かかっている。
        材料はSB450で、ヤング率は205[\si{\giga\pascal}]、降伏応力は250[\si{\mega\pascal}]である。 \\
        \Erase{本構造は薄肉構造と近似できるとする。} \Add{[後述]}
        \begin{itemize}
          \item[①]{円筒部のA点と半球部のB点の応力状態を求めよ}
          \item[②]{A点とB点ミーゼス相当応力を求め、降伏応力に達する臨界圧力を求めよ}
        \end{itemize}
    \end{column}
    \begin{column}{0.35\textwidth}
      \begin{figure}[htbp]
        \begin{center}
          \includegraphics[keepaspectratio,scale=2.2]{work/images/example-probrem.png}
            \caption{本日の例題(圧力容器)} \label{fig:example-probrem}
        \end{center}
      \end{figure}
    \end{column}
  \end{columns}
\end{frame}
