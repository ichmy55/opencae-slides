%
% ここから、各章毎に分割したソースファイルを順番に読込
%
\begin{document}
  %
  \maketitle                    % 表紙
  \input{101-abstract}          % はじめに
  \input{111-index-0}           % 目次
  %%%
  %%%
  \input{121-index-1}           % 
  \input{131-self-introduce}    % 自己紹介
  %%%
  \input{201-index-2}           % 
  \input{211-introduction}      % 
  %%%
  \input{301-index-3}           % 
  \begin{frame}{本日の例題}
 
    \begin{columns}[t]
    \begin{column}{0.65\textwidth}
        \\
        <参考文献\cite{wanted} より例題を拝借> \\
	    右\figurename \ref{fig:example-probrem}に示すような圧力容器に内圧10[\si{\mega\pascal}]が \\
        かかっている。
        材料はSB450で、ヤング率は205[\si{\giga\pascal}]、降伏応力は250[\si{\mega\pascal}]である。 \\
        \Erase{本構造は薄肉構造と近似できるとする。} \Add{[後述]}
        \begin{itemize}
          \item[①]{円筒部のA点と半球部のB点の応力状態を求めよ}
          \item[②]{A点とB点ミーゼス相当応力を求め、降伏応力に達する臨界圧力を求めよ}
        \end{itemize}
    \end{column}
    \begin{column}{0.35\textwidth}
      \begin{figure}[htbp]
        \begin{center}
          \includegraphics[keepaspectratio,scale=2.2]{work/images/example-probrem.png}
            \caption{本日の例題(圧力容器)} \label{fig:example-probrem}
        \end{center}
      \end{figure}
    \end{column}
  \end{columns}
\end{frame}
   % 本日の例題と以前の勉強会で報告された結果
  \input{312-previous-comments} % 以前の勉強会でのコメント
  %%%
  \input{401-index-4}           % 改めて解いてみた
  \begin{frame}{改めて解いてみた}
  \begin{itemize}
      \item[] 改めて解いてみた
      \begin{itemize}[itemsep=1.3ex, leftmargin=1cm]
        \item[(1)]  今回の改善点
	\item[(2)]  6面体メッシュのすすめ
	\item[(3)]  今回使用するソフト
	\item[(4)]  形状の作成
	\item[(5)]  反力のチェックサム
	\item[(6)]  表計算ソフトでグラフ化
      \end{itemize}
  \end{itemize}
\end{frame}
          % 改めて解いてみた
  \begin{frame}{今回の改善点}
 
    \begin{columns}[t]
    \begin{column}{0.7\textwidth}
        <今回の改善点> \\
         \begin{itemize}
            \item[①] コンター図がまだら模様の件 \\
                     これは4面体メッシュでの解析ではやむを得ない \\
                     \Add{6面体メッシュ}を用いてメッシュを切りなおす
            \item[]
            \item[②] 薄肉構造を仮定した件 \\
                     板厚方向に多くの分割数を確保し \\
                     板厚方向の\Add{応力分布}を\Add{厚肉円筒の応力式}と比較
            \item[]
            \item[③] メッシュ数の節約のため、\\
                     1/8ショートケーキモデルとする
         \end{itemize}
    \end{column}
    \begin{column}{0.3\textwidth}
      \begin{figure}[htbp]
        \begin{center}
          \includegraphics[keepaspectratio,scale=1.5]{images/previous.png}
            \caption{以前の勉強会で報告された結果(再掲)}
        \end{center}
      \end{figure}
    \end{column}
  \end{columns}
  %
  % TiKZを使った図形の描画 図2でA点を指し示す矢印
  \begin{textblock*}{30pt}(385pt,95pt)
    \begin{tikzpicture}
        \draw[->, draw=cud_red, line width=2pt] (0.7,0.8) -- (0,0.3);
        \node[rectangle,fill=cud_yellow,text width=0.5cm,text centered,rounded corners,minimum height=0.5cm](s) at (1cm,1cm) { \scriptsize A点};
    \end{tikzpicture}
  \end{textblock*}
\end{frame}
      % 今回の改善点
  \input{421-index-4C}          % 改めて解いてみた
  \input{422-meshing}           % 改めてメッシュについて復習
  \begin{frame}{改めて6面体要素の切り方について復習}
 \begin{table}[hbtp]
    \caption{6面体要素の切り方の例}
    \vspace{-5mm}
    \begin{NiceTabular}{|r|c|c|c|} % 表は項目名を右寄せ、データを中寄せ
       \hline
       名称       &  TransFinite法 & 押し出し法 &  スクリプト法 \\
       \midrule
       概要   & \begin{tabular}{c}形状をレンガ状に分割\\それぞれ構造格子\\を作る(曲面可)\end{tabular}
              & \begin{tabular}{c}平面メッシュを\\特定の方向に\\押し出す\end{tabular}
	      & \begin{tabular}{c}スクリプトで\\作成\\(特殊用途向け)\end{tabular} \\
       \hline
       概念図 & \includegraphics[keepaspectratio,height=15mm]{images/MappedCylinder.png}
              & \includegraphics[keepaspectratio,height=15mm]{images/sweep.png}
              & \includegraphics[keepaspectratio,height=15mm]{images/screw.png} \\
       \hline
       参考   & Wikipedia\cite{wiki}
	      & PrePoMax掲示板\cite{PrePoMax-news}
	      & ネジ解析の教科書\cite{fukuoka}\\
       \hline
       注意点 & \begin{tabular}{c}マルチブロック必須\\隣接ブロックと面を\\共有していないと割れ\end{tabular}
	      & 隣接との整合難
	      & アルゴリズム作成難 \\
       \hline
    \end{NiceTabular}
  \end{table}
\end{frame}
          % 改めてメッシュについて復習
  \input{431-index-4D}          % 
  \begin{frame}{今回使用するソフトと選定理由}
 \begin{table}[hbtp]
    \caption{今回使用するソフト}
    \begin{NiceTabular}{|c|c|c|c|c|c|} % 表は項目名を右寄せ、データを中寄せ
       \hline
	    用途     &  形状作成    & メッシュ切 & プリ & ソルバー & ポスト \\
       \hline
	    表で動作 &  gmsh        & \multicolumn{4}{|c|}{PrePoMax} \\
       \hline
	    内部動作 &  OpenCascade & gmsh       & PrePoMax  & Calculix & VTK \\
       \hline
    \end{NiceTabular}
  \end{table}
  \begin{itemize}
     \item[①] gmsh/OpenCascade :\\
              マルチブロックをまとめて一つのグループとして扱い \\
              ブロック間のメッシュの割れを防ぐ機能がある \\
              作成経過がスクリプト化され、形状の公開/後日の検証がしやすい
     \item[②] PrePoMax :
              とっつきやすそう
     \item[③] Calculix :
              上記の同梱だから
  \end{itemize}
\end{frame}
               % 今回使用するソフト
  \begin{frame}{改めて解いてみた}
  \begin{itemize}
      \item[] 改めて解いてみた
      \begin{itemize}[itemsep=1.3ex, leftmargin=1cm]
        \item[(1)]  {\color{cud_lightgray}今回の改善点}
	\item[(2)]  {\color{cud_lightgray}6面体メッシュのすすめ}
	\item[(3)]  {\color{cud_lightgray}今回使用するソフト}
	\item[▶(4)]   \highlight[cud_yellow]{形状の作成}
	\item[(5)]  {\color{cud_lightgray}反力のチェックサム}
	\item[(6)]  {\color{cud_lightgray}表計算ソフトでグラフ化}
      \end{itemize}
  \end{itemize}
\end{frame}
          % 
  \begin{frame}{形状作成}
  %
   \begin{columns}[t]
    \begin{column}{0.65\textwidth}
      \\
      今回はgmshの形状作成機能を使って、回転体で\\
      作ります \\
      ただし、今回の場合、形状に問題発生(後述) \\
      \vspace{5mm}
      のちほどgmshのセーブファイルを公開します\\
      ので、詳細の説明は省くものの、後で作成詳細を\\
      なぞることが可能です
    \end{column}
    \begin{column}{0.35\textwidth}
      \begin{figure}[htbp]
        \begin{center}
          \includegraphics[keepaspectratio,scale=2.2]{images/example-probrem.png}
            \caption{本日の例題(圧力容器・再掲)}
        \end{center}
      \end{figure}
    \end{column}
  \end{columns}
\end{frame}
      % 形状作成
  \begin{frame}{6面体メッシュが切れる形状の条件の復習}
  %
   \begin{columns}[t]
    \begin{column}{0.7\textwidth}
      <6面体メッシュが切れる形状の条件>
      \begin{itemize}
        \item[(1)] 4辺形6面で構成 \\
	           トポロジー的に直方体と等しいこと \\
		   (1面が縮退して5面体(プリズム状)も可能)
	\item[(2)] それぞれの辺/面は曲がっていてもよい
	\item[(3)] ただし、品質の良いメッシュを切るには、\\
		   角が90度に近いほうがよい
      \end{itemize}
    \end{column}
    \begin{column}{0.3\textwidth}
    \end{column}
  \end{columns}
\end{frame}
     % 形状作成
  \begin{frame}{今回の形状作成の概要}
  %
   \begin{columns}[t]
    \begin{column}{0.6\textwidth}
      <今回の形状作成の概要>
      \begin{itemize}
        \item[(1)]<1-> ます、側面の板を作って
	\item[(2)]<2-> 回転押し出し
	\item[(3)]<3-> レンガ状に切断
	\item[(4)]<4-> 回転中心軸に面積0の縮退面発生 \\
		       条件に違反のため6面体メッシュが \\
		       切れない
	\item[(5)]<5-> 左下の正常なブロックを回転コピー
	\item[(6)]<6-> 形状が完成したらbrep形式(Opencascade)
		       でExport
	\item[(7)]<6-> 詳細は配布物( ./01-gmsh-scriptから)
                       successful-examples.geo参照
      \end{itemize}
    \end{column}
    \begin{column}{0.4\textwidth}
      \vspace{-7mm}
      \begin{figure}[htbp]
        \begin{center}
	  \begin{overlayarea}{7cm}{15cm}
	    \only<1>{\includegraphics[keepaspectratio,scale=0.35]{images/sc1.png}}
	    \only<2>{\includegraphics[keepaspectratio,scale=0.35]{images/sc2.png}}
	    \only<3->{\includegraphics[keepaspectratio,scale=0.35]{images/sc3.png}}
            \caption{形状作成法の概要}
	  \end{overlayarea}
        \end{center}
      \end{figure}
    \end{column}
  \end{columns}
  \only<4>{
    \begin{textblock*}{30pt}(255pt,40pt)
      \begin{tikzpicture}
         \node[rectangle,fill=cud_yellow,text width=0.5cm,text centered,rounded corners,minimum height=0.5cm](s) at (1cm,1cm) { \scriptsize 面積0の縮退面};
         \draw[->, draw=cud_red, line width=1pt] (40pt,50pt) -- (133pt,70pt);
         \draw[draw=cud_red, line width=2pt] (133pt,66pt) -- (138pt,76pt);
      \end{tikzpicture}
    \end{textblock*}
  }
  \only<5->{
    \begin{textblock*}{30pt}(255pt,40pt)
      \begin{tikzpicture}
         \node[rectangle,fill=cud_yellow,text width=0.5cm,text centered,rounded corners,minimum height=0.5cm](s) at (1cm,1cm) { \scriptsize 回転コピ|};
         \draw[->, draw=cud_red, line width=1pt] (40pt,30pt) -- (95pt,0pt);
         \draw[->, draw=cud_red, line width=2pt] (103pt,-20pt) to [bend left=90]  (128pt,20pt);
      \end{tikzpicture}
    \end{textblock*}
  }
\end{frame}
     % 形状作成
  \begin{frame}{形状取り込み}
  %
   \begin{columns}[t]
    \begin{column}{0.6\textwidth}
      <今回の形状取り込みの概要>
      \begin{itemize}
        \item[(1)]<1-> PrePoMax を立ち上げ、
		       File→Newでまっさらなモデルを作成
        \item[(2)]<1-> 配布物( ./02-hmsh-shapeから)
		       successful-examples.brepをimport
        \item[(3)]<2-> すべての立体を選択し右クリック
                       「create compound part」で結合
        \item[(4)]<3-> 立体が1つの地続きの形状にまとまる
        \item[(5)]<4-> 元の立体の物性値やメッシュサイズを
		       個別に設定も可能となる
      \end{itemize}
    \end{column}
    \begin{column}{0.4\textwidth}
      \vspace{-7mm}
      \begin{figure}[htbp]
        \begin{center}
          \begin{overlayarea}{7cm}{15cm}
            \only<1>{\includegraphics[keepaspectratio,scale=0.30]{images/sc4.png}}
            \only<2>{\includegraphics[keepaspectratio,scale=0.30]{images/sc5.png}}
            \only<3->{\includegraphics[keepaspectratio,scale=0.30]{images/sc6.png}}
            \caption{形状取り込みの概要}
          \end{overlayarea}
        \end{center}
      \end{figure}
    \end{column}
  \end{columns}
  \only<1>{
    \begin{textblock*}{140pt}(210pt,60pt)
      \begin{tikzpicture}
         \node[rectangle,fill=cud_yellow,text width=70pt,text centered,rounded corners,minimum height=40pt](s) at (1cm,1cm) { \scriptsize 7つの立体が\\取り込まれる};
         \draw[->, draw=cud_red, line width=1pt] (10pt,50pt) -- (30pt,110pt);
      \end{tikzpicture}
    \end{textblock*}
  }
  \only<3>{
    \begin{textblock*}{100pt}(210pt,90pt)
      \begin{tikzpicture}
         \node[rectangle,fill=cud_yellow,text width=70pt,text centered,rounded corners,minimum height=40pt](s) at (1cm,1cm) { \scriptsize ひとかたまりの\\形状となる};
         \draw[->, draw=cud_red, line width=1pt] (20pt,50pt) -- (40pt,130pt);
      \end{tikzpicture}
    \end{textblock*}
  }
\end{frame}
      % 形状取り込み
  \begin{frame}{メッシュ切設定の概要}
  %
   \begin{columns}[t]
    \begin{column}{0.6\textwidth}
      <今回のメッシュ切設定の概要>
      \begin{itemize}
        \item[(1)]<1-> 全体のメッシュサイズを4mmに設定
        \item[(2)]<2-> 全体のメッシュを「TransFinite」に設定
	\item[(3)]<3-> 厚み方向のメッシュサイズを1mm(20分割)
		       に設定
        \item[(4)]<4-> メッシュ作成!
        \item[(5)]<5-> 板厚方向のメッシュが全立体に伝搬する
      \end{itemize}
    \end{column}
    \begin{column}{0.4\textwidth}
      \vspace{-7mm}
      \begin{figure}[htbp]
        \begin{center}
          \begin{overlayarea}{7cm}{15cm}
            \only<1>{\includegraphics[keepaspectratio,scale=0.30]{images/sc4.png}}
            \only<2>{\includegraphics[keepaspectratio,scale=0.30]{images/sc5.png}}
            \only<3->{\includegraphics[keepaspectratio,scale=0.30]{images/sc6.png}}
            \caption{メッシュ設定の概要}
          \end{overlayarea}
        \end{center}
      \end{figure}
    \end{column}
  \end{columns}
  \only<1>{
    \begin{textblock*}{140pt}(210pt,60pt)
      \begin{tikzpicture}
         \node[rectangle,fill=cud_yellow,text width=70pt,text centered,rounded corners,minimum height=40pt](s) at (1cm,1cm) { \scriptsize 7つの立体が\\取り込まれる};
         \draw[->, draw=cud_red, line width=1pt] (10pt,50pt) -- (30pt,110pt);
      \end{tikzpicture}
    \end{textblock*}
  }
  \only<3>{
    \begin{textblock*}{100pt}(210pt,90pt)
      \begin{tikzpicture}
         \node[rectangle,fill=cud_yellow,text width=70pt,text centered,rounded corners,minimum height=40pt](s) at (1cm,1cm) { \scriptsize ひとかたまりの\\形状となる};
         \draw[->, draw=cud_red, line width=1pt] (20pt,50pt) -- (40pt,130pt);
      \end{tikzpicture}
    \end{textblock*}
  }
\end{frame}
      % メッシュ切
  \begin{frame}{メッシュ切設定の概要}
  %
   \begin{columns}[t]
    \begin{column}{0.6\textwidth}
      <今回のメッシュ切設定の概要>
      \begin{itemize}
        \item[(5)]<1-> Mesh→Create Meshとすると \\
                       メッシュ作成する対象選択画面が現れる \\
                       Conpound-1に設定
        \item[(6)]<2-> 板厚方向のメッシュが全体に伝搬する
      \end{itemize}
    \end{column}
    \begin{column}{0.4\textwidth}
      \vspace{-7mm}
      \begin{figure}[htbp]
        \begin{center}
          \begin{overlayarea}{7cm}{15cm}
            \only<1>{\includegraphics[keepaspectratio,scale=0.30]{images/sc10.png}}
            \only<2>{\includegraphics[keepaspectratio,scale=0.30]{images/sc11.png}}
            \caption{メッシュ設定の概要}
          \end{overlayarea}
        \end{center}
      \end{figure}
    \end{column}
  \end{columns}
  \only<1>{
    \begin{textblock*}{160pt}(200pt,73pt)
      \begin{tikzpicture}
         \node[rectangle,fill=cud_yellow,text width=90pt,text centered,rounded corners,minimum height=40pt](s) at (1cm,1cm) { \scriptsize メッシュ作成対象};
         \draw[->, draw=cud_red, line width=1pt] (10pt,50pt) -- (80pt,100pt);
      \end{tikzpicture}
    \end{textblock*}
  }
  \only<2>{
    \begin{textblock*}{160pt}(200pt,90pt)
      \begin{tikzpicture}
         \node[rectangle,fill=cud_yellow,text width=90pt,text centered,rounded corners,minimum height=40pt](s) at (1cm,1cm) { \scriptsize  1か所の板厚方向の\\メッシュサイズ設定が\\全体に伝搬};
         \draw[->, draw=cud_red, line width=1pt] (50pt,50pt) -- (120pt,76pt);
      \end{tikzpicture}
    \end{textblock*}
  }
\end{frame}
      % メッシュ切
  \begin{frame}{境界条件作成}
   \begin{columns}[t]
    \begin{column}{0.6\textwidth}
      <今回の境界条件の概要>
      \begin{itemize}
        \item[(1)]<1-> 1/8ショートケーキモデルの \\
                       各対称面を拘束 
        \item[(2)]<2-> 続いて面圧を設定
      \end{itemize}
    \end{column}
    \begin{column}{0.4\textwidth}
      \vspace{-7mm}
      \begin{figure}[htbp]
        \begin{center}
          \begin{overlayarea}{7cm}{15cm}
            \only<1>{\includegraphics[keepaspectratio,scale=0.30]{images/sc12.png}}
            \only<2>{\includegraphics[keepaspectratio,scale=0.30]{images/sc13.png}}
            \caption{境界条件の概要}
          \end{overlayarea}
        \end{center}
      \end{figure}
    \end{column}
  \end{columns}
  \only<1>{
    \begin{textblock*}{160pt}(180pt,103pt)
      \begin{tikzpicture}
         \node[rectangle,fill=cud_yellow,text width=90pt,text centered,rounded corners,minimum height=20pt](s) at (1cm,3.0cm) { \scriptsize x方向固定};
         \draw[->, draw=cud_red, line width=1pt] (75pt,95pt) -- (125pt,130pt);
         \node[rectangle,fill=cud_yellow,text width=90pt,text centered,rounded corners,minimum height=20pt](s) at (2cm,1.2cm) { \scriptsize y方向固定};
         \draw[->, draw=cud_red, line width=1pt] (105pt,35pt) -- (125pt,40pt);
         \node[rectangle,fill=cud_yellow,text width=90pt,text centered,rounded corners,minimum height=20pt](s) at (7cm,0.0cm) { \scriptsize z方向固定};
         \draw[->, draw=cud_red, line width=1pt] (230pt,10pt) -- (225pt,90pt);
      \end{tikzpicture}
    \end{textblock*}
  }
	  \only<2>{
    \begin{textblock*}{160pt}(195pt,75pt)
      \begin{tikzpicture}
         \node[rectangle,fill=cud_yellow,text width=90pt,text centered,rounded corners,minimum height=40pt](s) at (1cm,0cm) { \scriptsize 面圧10MPa};
         \draw[->, draw=cud_red, line width=1pt] (10pt,20pt) -- (78pt,100pt);
         \draw[->, draw=cud_red, line width=1pt] (76pt,20pt) -- (130pt,30pt);
      \end{tikzpicture}
    \end{textblock*}
  }
\end{frame}
   % 境界条件作成
  \begin{frame}{もろもろ作成}
   \begin{columns}[t]
    \begin{column}{0.6\textwidth}
      <もろもろ作成の概要>
      \begin{itemize}
        \item[(1)]<1->  材質の作成 \\
		       「Material」を右クリックし「create」\\
		       「Elasticity」→「Elastic」を選択\\
		        ヤング率を206000[MPa]に \\
		        ポアソン比を0.3[-]に設定
        \item[(2)]<2->  続いて断面性状?の作成\\
                       「Sections」を右クリックし「create」\\
                       「Solid Section」を選択\\
                       「Material」を先ほど作成した「Material-1」に \\
                        対象にすべての立体を選択
      \end{itemize}
    \end{column}
    \begin{column}{0.4\textwidth}
      \vspace{-7mm}
      \begin{figure}[htbp]
        \begin{center}
          \begin{overlayarea}{7cm}{15cm}
            \only<1>{\includegraphics[keepaspectratio,scale=0.30]{images/sc14.png}}
            \only<2>{\includegraphics[keepaspectratio,scale=0.30]{images/sc15.png}}
            \caption{もろもろ作成の概要}
          \end{overlayarea}
        \end{center}
      \end{figure}
    \end{column}
  \end{columns}
  \only<1>{
    \begin{textblock*}{160pt}(200pt,70pt)
      \begin{tikzpicture}
         \node[rectangle,fill=cud_yellow,text width=90pt,text centered,rounded corners,minimum height=20pt](s) at (1cm,1cm) { \scriptsize 材質の作成};
         \draw[->, draw=cud_red, line width=1pt] (10pt,38pt) -- (42pt,110pt);
      \end{tikzpicture}
    \end{textblock*}
  }
  \only<2>{
    \begin{textblock*}{160pt}(195pt,42pt)
      \begin{tikzpicture}
         \node[rectangle,fill=cud_yellow,text width=90pt,text centered,rounded corners,minimum height=40pt](s) at (1cm,1cm) { \scriptsize Solid Section\\は 断面性状?};
         \draw[->, draw=cud_red, line width=1pt] (10pt,47pt) -- (55pt,114pt);
      \end{tikzpicture}
    \end{textblock*}
  }
\end{frame}
        % もろもろ作成
  \begin{frame}{こんなん出ましたけど??}
   \begin{columns}[t]
    \begin{column}{0.6\textwidth}
      <こんなん出ましたけど??>
      \begin{itemize}
        \item[(1)]<1-> JOBの設定は省略
        \item[(2)]<2-> 結果の表示 \\
		       まだらにならず、きれいな層状に表示\\
		       それっぽくは見えるが??\\
		       ここから詳しくチェック
      \end{itemize}
    \end{column}
    \begin{column}{0.4\textwidth}
      \vspace{-7mm}
      \begin{figure}[htbp]
        \begin{center}
          \begin{overlayarea}{7cm}{15cm}
            \only<2->{\includegraphics[keepaspectratio,scale=0.30]{images/sc16.png}
		      \caption{ミーゼス応力コンター図}}
          \end{overlayarea}
        \end{center}
      \end{figure}
    \end{column}
  \end{columns}
  \only<2>{
    \begin{textblock*}{160pt}(200pt,90pt)
      \begin{tikzpicture}
         \node[rectangle,fill=cud_yellow,text width=90pt,text centered,rounded corners,minimum height=20pt](s) at (1cm,1cm) { \scriptsize ミーゼス応力};
         \draw[->, draw=cud_red, line width=1pt] (10pt,38pt) -- (42pt,110pt);
      \end{tikzpicture}
    \end{textblock*}
  }
\end{frame}
         % 結果0
  %%%
  \input{501-index-5A}          % 改めて解いてみた
  \begin{frame}{反力のチェックサム(設定)}
 
  PrePoMaxにおいて、反力チェックサムを確認するには、以下設定する \\
  % 図の挿入
  \begin{figure}[htbp]
    \begin{center}
      \includegraphics[keepaspectratio,scale=0.285]{images/screen01.png}
      \caption{反力チェックサムの設定}
    \end{center}
  \end{figure}
\end{frame}
         % 結果1
  \begin{frame}{反力のチェックサム(結果)}
 
% PrePoMaxにおいて、反力チェックサムを確認するには、以下設定する \\

 % TiKZを使った図形の描画 図2でA点を指し示す矢印
  \begin{textblock*}{0.95\linewidth}(-10pt,-5pt)
    \begin{figure}[htbp]
      \begin{center}
        \begin{tikzpicture}
	  \node [above right,minimum height=50pt, minimum width=420pt,align=left] at (0pt,0pt) {PrePoMaxにおいて、反力チェックサムを確認するには、以下設定する};
          \node[rectangle,fill=cud_yellow,,above right,minimum height=70pt,minimum width=110pt,align=left] at (20pt,-60pt)
		{ \scriptsize{合計された結果を見るには}\\
		  \scriptsize{XX-REACTION-FORCEの}\\
		  \scriptsize{TOATALFORCE-RF1のタグを}\\
		  \scriptsize{確認する}};
          \node[above right,minimum height=170pt,minimum width=260pt,align=left] at (130pt,-160pt) {
	    \includegraphics[keepaspectratio,scale=0.265] {images/screen02.png}};
          \draw[->, draw=cud_red, line width=1pt] (120pt,-60pt) -- (170pt,-140pt);
          \node[rectangle,fill=cud_yellow,,above right,minimum height=70pt,minimum width=100pt,align=left] at (20pt,-160pt)
		{ \scriptsize{この値がX方向から見た}\\
		  \scriptsize{受圧面の投影面積に面圧}\\
		  \scriptsize{をかけたものが}\\
		  \scriptsize{等しいか確認する}\\
		  \scriptsize{今回は誤差0.5\%}};
          \draw[->, draw=cud_red, line width=1pt] (120pt,-120pt) -- (227pt,-35pt);
        \end{tikzpicture}
        \caption{反力チェックサムの確認}
      \end{center}
    \end{figure}
  \end{textblock*}
\end{frame}
         % 結果2
  \input{521-index-5B}          % 改めて解いてみた
  \input{522-results04}         % 結果3
  \input{523-results05}         % 結果4
  \input{524-results06}         % 結果5
  %%%
  \input{701-index-5}           % 
  \input{711-conclusions}       % まとめ
  %%%
  \begin{frame}{参考文献}
  % 参考文献リスト
   \beamertemplatetextbibitems
   \bibliographystyle{004-IEEEJtran}
   \bibliography{005-opencae}
\end{frame}
        % 参考文献
  %%%
  \begin{frame}{}
	\scalebox{2}{ご清聴、ありがとうございました}
\end{frame}
           % ご清聴、ありがとうございました
  \begin{frame}[noframenumbering]{付録A. ~今回使用した主なソフト~}
  \begin{table}[hbtp]
    \caption{今回使用した主なソフト}
    \vspace{-7mm}
    \begin{tabular}{|c||l|l|l|} \hline % 表は項目名を中央寄せ、データを左寄せ
	    用途          & 名称 & バージョン & URL \\ \hhline{|=:=|=|=|}
      3次元形状作成 & gmsh & 4.13.1 & {\urlstyle{same} \color{cud_orange}
                                   \href{https://gmsh.info}
				   {gmsh.info}}  \\ \hline
      プリ・ポスト  & PrePoMax & 2.2.3 & {\urlstyle{same} \color{cud_orange}
                                   \href{https://prepomax.fs.um.si/}
				   {prepomax.fs.um.si}}  \\ \hline
      ソルバ(PrePoMax同梱) & Calculix & 2.22 & {\urlstyle{same} \color{cud_orange}
                                   \href{https://calculix.de/}
				   {calculix.de}}  \\ \hline
      表計算        & LibreOffice & 24.8.0.3 & {\urlstyle{same} \color{cud_orange}
                                   \href{https://libreoffice.org/}
				   {libreoffice.org}}  \\ \hline
      OS            & Windows & 11Pro(24H2) & {\urlstyle{same} \color{cud_orange}
                                   \href{https://www.microsoft.com/}
				   {microsoft.com}}  \\ \hline
    \end{tabular}
    \\(注:各ソフトはWindows版x86\_64/amd64用を使用しています)
  \end{table}
\end{frame}
      % 付録 ~今回使用したソフト~
  \input{912-appendices-B}      % 付録 ~配布するデータファイル~
  \begin{frame}[noframenumbering]{付録C. ~配布するデータファイル~}
  前ページで示した配布物の中身は以下のとおり
   \begin{table}[hbtp]
    \caption{Handout.zipの中身}
    \vspace{-2mm}
   {\scriptsize
      \begin{tabular}{|c||l|l|l|} \hline % 表は項目名を中央寄せ、データを左寄せ
        ディレクトリ & ファイル種類 & ファイル名 & 概要 \\ \hhline{|=:=|=|=|}
	\multirow{2}{*}{01-gmsh-script} & gmsh入力  & failure-example.geo & メッシュ切失敗形状  \\ \cline{3-4}
				        & スクリプト& successful-example.geo & メッシュ切成功形状  \\ \hline
        \multirow{4}{*}{02-gmsh-shape}  & gmsh形状出力  & failure-example.brep & メッシュ切失敗形状  \\ \cline{3-4}
				        & B-rep形式 & successful-example.brep & メッシュ切成功形状  \\ \cline{2-4}
                                        & gmsh形状出力  & failure-example.step & メッシュ切失敗形状  \\ \cline{3-4}
                                        & step形式  & successful-example.step & メッシュ切成功形状  \\ \hline
        03-prepomax-savefile            & prepomaxセーブ  & success-example-hex8.pmx &   \\ \hline
        04-calculix-input               & calculix入力 & success-example-hex8.inp &   \\ \hline
   \multirow{2}{*}{05-caluculix-output} & \multirow{2}{*}{calculix出力}  & success-example-hex8.frd & 全接点応力出力  \\ \cline{3-4}
				        & & success-example-hex8.dat & 特定接点変位出力  \\ \hline
        06-libreoffice                  & 表計算 & successful-example-hex8.ods &   \\ \hline
      \end{tabular}
    }
  \end{table}
\end{frame}
      % 付録 ~ソースのありか~
\end{document}
